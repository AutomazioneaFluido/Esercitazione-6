\documentclass[a4paper]{article}
\usepackage[T1]{fontenc}
\usepackage[utf8x]{inputenc}
\usepackage[italian,english]{babel}
\usepackage{amssymb,latexsym,amsfonts,amsmath}
\usepackage{lipsum}
\usepackage{url}
\usepackage{graphicx}
\usepackage[enable-survey]{pdfpages}

\begin{document}


\title{Esercitazione 6}
\date{April 5 , 2017}
\maketitle


\author{Alessio Susco \hspace*{2cm} Nicola Bomba \hspace*{2cm} Fabrizio Ursini  \\  \hspace*{1,85cm} Alessandra Di Martino \hspace*{1,25cm} Diego Guzman}

\includepdf[pages={1},pagecommand={\thispagestyle{plain}}]{eserc6.pdf} 

\tableofcontents

\clearpage

\section{Introduzione Generale}
L’obiettivo della prova è quello di realizzare un circuito di comando di un sistema elettropneumatico che coinvolga 2 o 3 cilindri pneumatici utilizzando le funzioni di temporizzazione e conteggio. Includiamo poi nella relazione:

\begin{itemize}
\item Il diagramma movimento-fasi;
\item Il grafcet;
\item Il programma in rete ladder.
Si devono utilizzare due diversi PLC: OMRON e SIEMENS.
\end{itemize}

\section{Strumenti Utilizzati}

\subsection{Prova con SIEMENS}
Banco degli attuatori, che comprende:
\begin{itemize}
\item Cilindri pneumatici a doppio effetto x3;
\item Valvole bistabili a comando elettropneumatico x3;
\item Valvole monostabili di fine corsa a comando elettropneumatico x6;
\item Tubi in poliuretano;
\item Cavi elettrici;
\item Alimentazione (aria compressa);
\item Alimentazione elettrica 24V.
\end{itemize}

Banco del PLC, che comprende:
\begin{itemize}
\item Valvola monostabile a pulsante;
\item Valvola bistabile a leva;
\item Valvola monostabile a pulsante di emergenza;
\item Lampadine elettriche x2;
\item Lampadine pneumatiche x2;
\item Valvola monostabile a comando elettropneumatico;
\item Valvola bistabile a comando elettropneumatico;
\item Switch di accensione/spegnimento;
\item PLC SIEMENS.
\end{itemize}
E' stato infine utilizzato un Computer con il Software STEP 7 per la programmazione Ladder del PLC SIEMENS.


\subsection{Prova con OMRON}
Banco degli attuatori, che comprende:
\begin{itemize}
\item Cilindri pneumatici a doppio effetto x3;
\item Valvole bistabili a comando elettropneumatico x3;
\item Valvole monostabili di fine corsa a comando elettropneumatico x6;
\item Tubi in poliuretano;
\item Cavi elettrici;
\item Alimentazione (aria compressa);
\item Alimentazione elettrica 24V.
\end{itemize}

Banco del PLC, che comprende:
\begin{itemize}
\item Valvola monostabile a pulsante;
\item Valvola bistabile a leva;
\item Valvola monostabile a pulsante di emergenza;
\item Lampadine elettriche x2;
\item Lampadine pneumatiche x2;
\item Valvola monostabile a comando elettropneumatico;
\item Valvola bistabile a comando elettropneumatico;
\item Switch di accensione/spegnimento;
\item PLC OMRON.
\end{itemize}

E' stato infine utilizzato un Computer con il Software SYSWIN per la programmazione Ladder del PLC OMRON.

\section{Osservazione Preliminare}
\subsection{Prova con SIEMENS}
\subsubsection{Prova 0: Esercitazione alla Lavagna}
Come prova preliminare effettuiamo un semplice esercizio di uscita e rientro di un cilindro A. Nel programma riportiamo i comandi di input, output e i marker. Di seguito riportiamo gli schemi necessari.
\subsubsection{Prova 1: Esercitazione 5.1}
Riprendiamo lo schema del primo esercizio dell’esercitazione 5, in cui utilizziamo due cilindri A e B. Scriviamo grafcet, grafcet contratto, rete ladder, diagramma movimento-fasi e equazioni logiche, e successivamente riportiamo nel programma OMRON i comandi.
\subsubsection{Prova 2: Esercitazione 5.2}
In questo caso prendiamo lo schema del secondo esercizio dell’esercitazione 5, e ripetiamo gli step della prova precedente.
\subsection{Prova con OMRON}
\subsubsection{Prova 0: Esercitazione alla Lavagna}
Basandoci sulla prova 0 del PLC SIEMENS usiamo lo stesso schema ma aggiungiamo un temporizzatore nel momento in cui il cilindro è fuoriuscito. Per programmare il PLC SIEMENS usiamo un linguaggio diverso, in cui input, output e marker sono rappresentati con altre simbologie. Nella documentazione includiamo anche le reti ladder secondo il progetto strutturato di comando e di azionamento.
\subsubsection{Prova 1: Esercitazione 5.1}
Sempre con lo schema della prova 1 ora aggiungiamo un contatore dopo il rientro del cilindro A, che determinerà il numero di cicli da effettuare prima di interrompersi automaticamente. Inoltre includiamo tutta la documentazione elencata precedentemente.
\section{Schema Circuito}
\subsection{Schema Esercizio 1}

\begin{center}
%\includegraphics[scale=0.6]{ES4ES1P2.png}
\end{center}

\subsection{Schema Esercizio 2}
\begin{center}
%\includegraphics[scale=0.6]{ES4ES2P1.png}
\end{center}


\section{Calcoli}
\dots
\clearpage

\section{Grafici Esercizio 1}

\subsection{Diagramma movimento-fasi}
\subsection{Grafcet Contratto}
Grafcet relativo all'esercizio con Temporizzatore
\begin{center}
\includegraphics[scale=0.6]{GrafcetES1.png}
\end{center}
\subsection{Programmazione Strutturata in Rete Ladder}
\subsubsection{Inizializzazione}
\subsubsection{Gestione Ciclo Automatico}
\begin{center}
\includegraphics[scale=0.6]{GesAzES1.png}
\end{center}
\subsubsection{Esecuzione Azioni}
\begin{center}
\includegraphics[scale=0.6]{EsAzES1.png}
\end{center}
\subsubsection{Gestione Allarmi}
Specificare nelle osservazioni che è stata ignorata la parte sulla gestione allarmi
\clearpage

\section{Grafici Esercizio 2}

\subsection{Diagramma movimento-fasi}
\subsection{Grafcet Contratto}
Grafcet relativo all'esercizio con Temporizzatore e Contatore
\begin{center}
\includegraphics[scale=0.6]{GrafcetES2.png}
\end{center}

\subsection{Programmazione Strutturata in Rete Ladder}
\subsubsection{Inizializzazione}
\subsubsection{Gestione Ciclo Automatico}
\begin{center}
%\includegraphics[scale=0.6]{GesAzES1.png}
\end{center}
\subsubsection{Esecuzione Azioni}
\begin{center}
%\includegraphics[scale=0.6]{EsAzES1.png}
\end{center}
\subsubsection{Gestione Allarmi}
Specificare nelle osservazioni che è stata ignorata la parte sulla gestione allarmi
\clearpage






\section{Descrizione Approfondita dell'Esercitazione}
\subsection{Descrizione Esercizio 1}
...
\begin{itemize}
\item ...
\end{itemize}
...

\subsection{Descrizione Esercizio 2}
\dots

\section{Conclusioni}
\subsection{Conclusioni Esercizio 1}
\dots
\subsection{Conclusioni Esercizio 2}
\dots


\end{document}
